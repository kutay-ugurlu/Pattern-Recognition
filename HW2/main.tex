\documentclass[11pt]{extarticle}
\usepackage{mathtools}
\usepackage[a4paper, total={6in, 8.5in}]{geometry}
\usepackage{graphicx}
\usepackage{subfig}
\usepackage{amssymb}
\usepackage{amsmath}
\usepackage{pythonhighlight}
\usepackage{pdfpages}
\usepackage[T1]{fontenc}
\usepackage[utf8]{inputenc}
\usepackage{fancyhdr}
\usepackage{pythonhighlight}
\usepackage{changepage}
\usepackage{slashbox}
\usepackage{floatrow}
\usepackage{subfig}
\usepackage{listings}
\usepackage{color} %red, green, blue, yellow, cyan, magenta, black, white
\definecolor{mygreen}{RGB}{28,172,0} % color values Red, Green, Blue
\definecolor{mylilas}{RGB}{170,55,241}



\floatsetup[table]{capposition=top}

\sloppy
\definecolor{lightgray}{gray}{0.5}
\setlength{\parindent}{0pt}
\setlength{\headheight}{14pt}

\renewcommand{\headrulewidth}{.4mm} % header line width
\newcommand{\norm}[1]{\left\lVert#1\right\rVert}


\pagestyle{fancy}
\fancyhf{}
\fancyhfoffset[L]{-1cm} % left extra length
\fancyhfoffset[R]{-1cm} % right extra length
\rhead{\bfseries Kutay U\u{g}urlu 2232841}
\lhead{EE583 Homework 2}
\rfoot{}

\DeclarePairedDelimiter\ceil{\lceil}{\rceil}
\DeclarePairedDelimiter\floor{\lfloor}{\rfloor}

\author{Kutay U\u{g}urlu 2232841}

\begin{document}

\lstset{language=Matlab,%
    %basicstyle=\color{red},
    breaklines=true,%
    morekeywords={matlab2tikz},
    keywordstyle=\color{blue},%
    morekeywords=[2]{1}, keywordstyle=[2]{\color{black}},
    identifierstyle=\color{black},%
    stringstyle=\color{mylilas},
    commentstyle=\color{mygreen},%
    showstringspaces=false,%without this there will be a symbol in the places where there is a space
    numbers=left,%
    numberstyle={\tiny \color{black}},% size of the numbers
    numbersep=9pt, % this defines how far the numbers are from the text
    emph=[1]{for,end,break},emphstyle=[1]\color{red}, %some words to emphasise
    %emph=[2]{word1,word2}, emphstyle=[2]{style},    
}

\fancyfoot[C]{\thepage}
\title{\LARGE \LARGE EE583 Pattern Recognition HW2}

\maketitle{\LARGE}

\pagebreak
\section*{Question 1}

Maximum Likelihood Estimate of Multivariate Gaussian random variable are calculated as:
\begin{itemize}
    \item $\hat{\mu} = \frac{1}{m}\sum_{i=1}^mX^{(i)}$
    \item $\hat{\Sigma} = \frac{1}{m}\sum_{i=1}^m(X^{(i)}-\hat{\mu})(X^{(i)}-\hat{\mu})^T$
\end{itemize}
where $X^{(i)}$ is the $i^{th}$ observation.\\

\lstinputlisting{HW2_Q1.m}

The given MATLAB code produces the following results: \\
\begin{minipage}{0.2\textwidth}
    $m = 10$
\end{minipage}
\begin{minipage}{0.35\textwidth}
    $\hat{\mu} = \begin{bmatrix} -0.4519 & 1.0984 \end{bmatrix}$
\end{minipage}
\begin{minipage}{0.66\textwidth}
    $\hat{\Sigma} = \begin{bmatrix} 0.3191 & 0.1799 \\ 0.1799 & 0.6506 \end{bmatrix}$
\end{minipage}
\begin{minipage}{0.2\textwidth}
    $m = 1000$
\end{minipage}
\begin{minipage}{0.35\textwidth}
    $\hat{\mu} = \begin{bmatrix} -0.7367 & 0.5176 \end{bmatrix}$
\end{minipage}
\begin{minipage}{0.66\textwidth}
    $\hat{\Sigma} = \begin{bmatrix} 0.4730 & 0.2925 \\ 0.2925 & 0.8079 \end{bmatrix}$
\end{minipage}
It is observed that the estimations get more accurate with increasing number of samples.

\section*{Question 2}

\subsection*{25 Samples}
\begin{minipage}{0.3\textwidth}
    ${}\hat{x}}_{ML} = 2.7456$
\end{minipage}
\begin{minipage}{0.6\textwidth}
    ${}\hat{x}}_{MAP} = 2.7475$
\end{minipage}

\subsection*{1000 samples}
\begin{minipage}{0.3\textwidth}
    ${}\hat{x}}_{ML} = 2.7456$
\end{minipage}
\begin{minipage}{0.6\textwidth}
    ${}\hat{x}}_{MAP} = 2.7475$
\end{minipage}


\end{document}